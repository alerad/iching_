\documentclass[11pt]{article}
\usepackage[utf8]{inputenc}
\usepackage[T1]{fontenc}
\usepackage{lmodern}
\usepackage{geometry}
\geometry{margin=1in}
\usepackage{amsmath,amssymb,amsthm}
\usepackage{microtype}
\usepackage{booktabs}
\usepackage{tikz}
\usetikzlibrary{positioning}
\usepackage[colorlinks=true,
  linkcolor=blue!60!black,
  citecolor=teal!60!black,
  urlcolor=magenta!70!black]{hyperref}

% theorem environments
\theoremstyle{definition}
\newtheorem{definition}{Definition}[section]
\newtheorem{remark}[definition]{Remark}
\newtheorem{example}[definition]{Example}
\theoremstyle{plain}
\newtheorem{theorem}[definition]{Theorem}
\newtheorem{corollary}[definition]{Corollary}
\newtheorem{lemma}[definition]{Lemma}
\newtheorem{proposition}[definition]{Proposition}

% macros
\newcommand{\R}{\mathbb{R}}
\newcommand{\Z}{\mathbb{Z}}
\newcommand{\B}{\{0,1\}}
\newcommand{\Ham}{d_H}
\newcommand{\comp}{\mathrm{comp}}
\newcommand{\rev}{\mathrm{rev}}

\title{\textbf{Optimal Equivariant Matchings on the 6-Cube}\\[0.3em]
\large With an Application to the King Wen Sequence}
\author{Alejandro Radisic}
\date{\today}

\begin{document}
\maketitle

\begin{abstract}
We characterize perfect matchings on the Boolean hypercube $\B^n$ that are equivariant under the Klein four-group $K_4 = \langle \comp, \rev \rangle$ generated by bitwise complement and reversal. For $n = 6$, we prove there exists a \emph{unique} $K_4$-equivariant matching minimizing total Hamming cost among matchings using only $\comp$ or $\rev$ pairings, achieving cost 120 versus 192 for the complement-only matching. The optimal matching is determined by a simple ``reverse-priority rule'': pair each element with its reversal unless it is a palindrome, in which case pair with its complement. We verify that the historically significant King Wen sequence of the I Ching is isomorphic to this optimal matching. Notably, allowing $\comp \circ \rev$ pairings yields lower cost (96), but the King Wen sequence follows the structurally simpler rule. All results are formally verified in Lean~4 with the Mathlib library.
\end{abstract}

\section{Introduction}

The Boolean hypercube $\B^n = \{0,1\}^n$ admits a natural action by the Klein four-group $K_4 \cong \Z_2 \times \Z_2$, generated by bitwise complement $\comp$ and bit reversal $\rev$. A \emph{$K_4$-equivariant perfect matching} is a partition of $\B^n$ into pairs such that paired elements lie in the same $K_4$-orbit.

This paper addresses the following optimization problem:
\begin{quote}
\emph{Among $K_4$-equivariant perfect matchings on $\B^n$ where each pair uses either $\comp$ or $\rev$, which minimize total Hamming distance?}
\end{quote}

For $n = 6$, we prove uniqueness: there is exactly one optimal matching of this form, determined by a simple greedy rule. As an application, we verify that the King Wen sequence---a traditional ordering of the 64 hexagrams of the I Ching into 32 pairs---is isomorphic to this unique optimal matching. We also note that relaxing the constraint to allow $\comp \circ \rev$ pairings yields a lower-cost matching (96 vs 120), but one without the structural elegance of the reverse-priority rule.

\subsection{Main Results}

\begin{theorem}[Uniqueness of Optimal $\comp/\rev$ Matching]\label{thm:main}
For $n = 6$, there exists a unique $K_4$-equivariant perfect matching on $\B^6$ minimizing total Hamming distance among matchings where each pair is either $\{h, \comp(h)\}$ or $\{h, \rev(h)\}$. This matching is given by the \emph{reverse-priority rule}:
\[
\mathrm{partner}(h) =
\begin{cases}
\rev(h) & \text{if } h \neq \rev(h) \\
\comp(h) & \text{if } h = \rev(h)
\end{cases}
\]
The total Hamming cost is $120$, compared to $192$ for the complement-only matching.
\end{theorem}

\begin{theorem}[Isomorphism with King Wen]\label{thm:iching}
The King Wen sequence of the I Ching defines a perfect matching on $\B^6$ that is isomorphic to the unique optimal $K_4$-equivariant matching of Theorem~\ref{thm:main}.
\end{theorem}

\subsection{Formal Verification}

All theorems are formally verified in Lean~4 with the Mathlib library. Key modules:
\begin{itemize}
    \item \texttt{IChing/Hexagram.lean}: $\B^6$ representation, Hamming distance
    \item \texttt{IChing/Symmetry.lean}: $K_4$-action, orbit structure
    \item \texttt{IChing/KingWenOptimality.lean}: Optimality and uniqueness proofs
\end{itemize}

\section{Preliminaries}

\subsection{Hexagrams as Binary Vectors}

\begin{definition}[Hexagram]
A \emph{hexagram} is an element $h \in \B^6 = \{0,1\}^6$. We index positions $0$ (bottom) through $5$ (top), following the traditional convention that hexagrams are read from bottom to top. The correspondence between hexagrams and binary numbers was noted by Leibniz~\cite{leibniz1703}.
\end{definition}

There are $2^6 = 64$ hexagrams, corresponding to the vertices of the 6-dimensional hypercube.

\begin{definition}[Hamming Distance]
For $h_1, h_2 \in \B^6$, the \emph{Hamming distance} is
\[
\Ham(h_1, h_2) = \#\{i : h_1(i) \neq h_2(i)\} = \sum_{i=0}^{5} |h_1(i) - h_2(i)|
\]
\end{definition}

\begin{definition}[Total Hamming Cost of a Matching]
For a perfect matching $M$ on $\B^6$ (a partition into 32 disjoint pairs), the \emph{total Hamming cost} is the sum of Hamming distances over all pairs:
\[
\mathrm{Cost}(M) = \sum_{\{h_1, h_2\} \in M} \Ham(h_1, h_2)
\]
This measures how ``different'' the paired elements are overall.
\end{definition}

\subsection{The Klein Four-Group Action}

\begin{definition}[Complement and Reversal]
Define operations on hexagrams:
\begin{align*}
\comp(h)(i) &= 1 - h(i) \quad \text{(bitwise complement)} \\
\rev(h)(i) &= h(5-i) \quad \text{(bit reversal)}
\end{align*}
\end{definition}

\begin{proposition}\label{prop:klein}
The operations $\comp$ and $\rev$ satisfy:
\begin{enumerate}
    \item $\comp \circ \comp = \mathrm{id}$ (complement is an involution)
    \item $\rev \circ \rev = \mathrm{id}$ (reversal is an involution)
    \item $\comp \circ \rev = \rev \circ \comp$ (they commute)
\end{enumerate}
Thus $\{\mathrm{id}, \comp, \rev, \comp \circ \rev\}$ forms the Klein four-group $\Z_2 \times \Z_2$.
\end{proposition}

\begin{definition}[Orbit]
The \emph{orbit} of a hexagram $h$ under the Klein four-group is
\[
\mathrm{orbit}(h) = \{h, \comp(h), \rev(h), \comp(\rev(h))\}
\]
\end{definition}

\begin{proposition}[Orbit Sizes]
For any hexagram $h$, $|\mathrm{orbit}(h)| \in \{2, 4\}$.
\begin{itemize}
    \item Size 4: generic hexagrams (48 total, forming 12 orbits)
    \item Size 2: palindromes with $h = \rev(h)$ (8 total, forming 4 orbits)
    \item Size 2: anti-symmetric hexagrams with $\rev(h) = \comp(h)$ but $h \neq \rev(h)$ (8 total, forming 4 orbits)
\end{itemize}
\end{proposition}

\begin{definition}[Anti-Symmetric Hexagram]
A hexagram $h$ is \emph{anti-symmetric} if $h(i) \neq h(5-i)$ for all $i \in \{0,1,2\}$. Equivalently, $\rev(h) = \comp(h)$.
\end{definition}

\begin{definition}[Palindrome]
A hexagram $h$ is a \emph{palindrome} if $\rev(h) = h$.
\end{definition}

\subsection{Hamming Distances of Group Actions}

\begin{proposition}\label{prop:distances}
For any hexagram $h$:
\begin{enumerate}
    \item $\Ham(h, \comp(h)) = 6$ (complement flips all bits)
    \item $\Ham(h, \rev(h)) \leq 6$, with equality iff $h(i) \neq h(5-i)$ for all $i$
\end{enumerate}
\end{proposition}

\begin{lemma}[Key Lemma]\label{lem:key}
If $\Ham(h, \rev(h)) = 6$, then $\rev(h) = \comp(h)$.
\end{lemma}

\begin{proof}
If all 6 positions differ between $h$ and $\rev(h)$, then for each $i$: $\rev(h)(i) = h(5-i) \neq h(i)$. Since bits are in $\{0,1\}$, this means $h(5-i) = 1 - h(i) = \comp(h)(i)$. Thus $\rev(h) = \comp(h)$.
\end{proof}

This lemma is crucial: when reversal achieves maximum distance 6, it coincides with complement, so there is no choice to make.

\section{The King Wen Sequence}

\subsection{Definition}

The King Wen sequence assigns each hexagram a number 1--64 according to tradition~\cite{wilhelm1967}. We represent it via a mapping from King Wen numbers to binary:

\begin{definition}[King Wen Binary Mapping]
Define $\mathrm{KW} : \{0, \ldots, 63\} \to \B^6$ as follows (index 0 corresponds to King Wen number 1):
\begin{center}
\small
\begin{tabular}{r|cccccccc}
KW & 1 & 2 & 3 & 4 & 5 & 6 & 7 & 8 \\
Binary & 63 & 0 & 17 & 34 & 23 & 58 & 2 & 16 \\
\midrule
KW & 9 & 10 & 11 & 12 & 13 & 14 & 15 & 16 \\
Binary & 55 & 59 & 7 & 56 & 61 & 47 & 4 & 8 \\
\end{tabular}
\end{center}
(Full table in Appendix.)
\end{definition}

\begin{definition}[King Wen Pairs]
The \emph{King Wen pairs} are $(h_{2k}, h_{2k+1})$ for $k = 0, \ldots, 31$, where $h_i = \mathrm{KW}(i)$.
\end{definition}

\subsection{The Equivariance Theorem}

\begin{theorem}[100\% Structural Regularity]\label{thm:equivariant}
Every King Wen pair $(h_1, h_2)$ satisfies:
\[
h_2 = \comp(h_1) \quad \text{or} \quad h_2 = \rev(h_1)
\]
The 32 pairs partition as:
\begin{itemize}
    \item 4 pairs: palindromes, paired by complement ($h_2 = \comp(h_1)$, distance 6)
    \item 4 pairs: anti-symmetric, paired by reversal $=$ complement ($h_2 = \rev(h_1) = \comp(h_1)$, distance 6)
    \item 24 pairs: generic, paired by reversal ($h_2 = \rev(h_1)$, distance 2 or 4)
\end{itemize}
\end{theorem}

\begin{proof}
Verified computationally over all 32 pairs. In Lean 4: \texttt{decide}.
\end{proof}

\begin{corollary}
The King Wen pairing respects the Klein four-group: paired hexagrams always lie in the same orbit.
\end{corollary}

\section{Optimality of the King Wen Matching}

\subsection{The Reverse-Priority Rule}

\begin{definition}[Reverse-Priority Matching]
Define the \emph{priority partner} function:
\[
\mathrm{partner}(h) =
\begin{cases}
\comp(h) & \text{if } h = \rev(h) \text{ (palindrome)} \\
\rev(h) & \text{otherwise}
\end{cases}
\]
\end{definition}

The intuition: prefer reversal (which has distance $\leq 6$) unless reversal is trivial (palindrome), in which case use complement.

\begin{theorem}[Involution]\label{thm:involution}
The priority partner function is an involution: $\mathrm{partner}(\mathrm{partner}(h)) = h$.
\end{theorem}

\begin{proof}
Two cases:
\begin{enumerate}
    \item If $h$ is a palindrome, then $\comp(h)$ is also a palindrome (complement preserves palindrome structure), so $\mathrm{partner}(\comp(h)) = \comp(\comp(h)) = h$.
    \item If $h$ is not a palindrome, then $\rev(h)$ is also not a palindrome, so $\mathrm{partner}(\rev(h)) = \rev(\rev(h)) = h$.
\end{enumerate}
\end{proof}

\subsection{Cost Minimization}

\begin{definition}[Total Hamming Cost]
For a perfect matching $M$ on $\B^6$, the \emph{total Hamming cost} is
\[
\mathrm{Cost}(M) = \sum_{\{h_1, h_2\} \in M} \Ham(h_1, h_2)
\]
\end{definition}

\begin{theorem}[Optimality among $\comp/\rev$ Matchings]\label{thm:optimal}
Among $K_4$-equivariant perfect matchings on $\B^6$ where each pair uses $\comp$ or $\rev$, the reverse-priority matching uniquely minimizes total Hamming cost.
\end{theorem}

\begin{proof}
Restricting to $\comp/\rev$ pairings, the options for each hexagram $h$ are $\comp(h)$ or $\rev(h)$.
\begin{itemize}
    \item By Proposition~\ref{prop:distances}, $\Ham(h, \rev(h)) \leq 6 = \Ham(h, \comp(h))$.
    \item Equality holds iff $\rev(h) = \comp(h)$ (Lemma~\ref{lem:key}).
    \item For palindromes, $\rev(h) = h$, so the only non-trivial option is $\comp(h)$.
\end{itemize}
Thus the reverse-priority rule chooses the minimum-distance option at each step. Since there is no actual choice when distances are equal (the options coincide), the matching is uniquely determined.
\end{proof}

\begin{remark}[The $\comp \circ \rev$ Alternative]\label{rem:cr}
On a size-4 orbit, a third equivariant pairing exists: $\{h, \comp(\rev(h))\}$. For orbits where $\Ham(h, \rev(h)) = 4$, we have $\Ham(h, \comp(\rev(h))) = 2$, making the $\comp \circ \rev$ pairing strictly cheaper. A mixed strategy using $\comp \circ \rev$ on such orbits achieves total cost $96 < 120$. However, this requires case-by-case analysis of each orbit, whereas the reverse-priority rule is a uniform structural principle. The King Wen sequence follows the elegant rule rather than the raw optimum.
\end{remark}

\begin{corollary}[Cost Values]
The total Hamming cost of the reverse-priority matching:
\begin{itemize}
    \item Palindrome pairs: $4 \times 6 = 24$
    \item Anti-symmetric pairs: $4 \times 6 = 24$
    \item Generic pairs: $12 \times 2 + 12 \times 4 = 72$
\end{itemize}
Total: $24 + 24 + 72 = 120$.

For comparison, the complement-only matching has cost $32 \times 6 = 192$.
\end{corollary}

\subsection{Uniqueness}

\begin{theorem}[Uniqueness]\label{thm:unique}
Any equivariant matching satisfying the reverse-priority rule equals the priority partner function.
\end{theorem}

\begin{proof}
The reverse-priority rule uniquely specifies the partner at each hexagram. By Theorem~\ref{thm:involution}, this defines a valid involution, hence a perfect matching.
\end{proof}

\begin{corollary}[King Wen is Canonical]
The King Wen sequence encodes the unique cost-minimizing $\comp/\rev$ equivariant matching on $\B^6$.
\end{corollary}

\section{Discussion}

\subsection{Orbit Structure}

The $K_4$-action on $\B^6$ partitions the 64 elements into orbits of size 2 or 4:
\begin{itemize}
    \item \textbf{Size-4 orbits}: 12 orbits containing 48 generic elements
    \item \textbf{Size-2 orbits}: 8 orbits containing 16 palindromes (elements fixed by $\rev$)
\end{itemize}

Within each orbit, the equivariant matching must pair elements. For size-4 orbits, there are three equivariant pairings induced by the non-identity involutions $\comp$, $\rev$, and $\comp \circ \rev$. The reverse-priority rule uses $\rev$ (falling back to $\comp$ for palindromes), which minimizes cost among $\comp/\rev$ matchings. A mixed strategy using $\comp \circ \rev$ on orbits where $\Ham(h, \rev(h)) = 4$ achieves the global minimum cost of 96.

\subsection{Uniqueness Mechanism}

The key to uniqueness is Lemma~\ref{lem:key}: when $\Ham(h, \rev(h)) = 6$, we have $\rev(h) = \comp(h)$, so the two candidate partners coincide. This eliminates all apparent degrees of freedom in the optimization.

\subsection{Formal Verification}

All results are machine-checked in Lean~4 using the Mathlib library:
\begin{itemize}
    \item \texttt{decide} tactic for computational verification of all 32 pairs
    \item Constructive proof that the reverse-priority function is an involution
    \item Proof of optimality via case analysis on orbit structure
\end{itemize}

\subsection{Subgroup Matchings}

One may ask whether the theorem holds for subgroups of $K_4$:
\begin{itemize}
    \item \textbf{Complement-only} ($\langle \comp \rangle$): Valid perfect matching with cost 192, trivially unique since each hexagram has exactly one complement.
    \item \textbf{Reversal-only} ($\langle \rev \rangle$): \emph{Not} a valid perfect matching. The 8 palindromes satisfy $\rev(h) = h$, so they cannot pair with their own reversal. A perfect matching requires pairing with a \emph{different} element.
\end{itemize}
Thus the reverse-priority rule is the minimal hybrid that produces a valid matching: use $\rev$ whenever possible, fall back to $\comp$ for the 8 palindromes.

\subsection{Extensions}

The $K_4$-action generalizes to $\B^n$ for any $n$. The orbit structure and optimal matching problem remain well-defined; we conjecture the reverse-priority rule remains optimal for all $n$.

\begin{thebibliography}{9}

\bibitem{wilhelm1967}
Richard Wilhelm and Cary F. Baynes.
\newblock {\em The I Ching or Book of Changes}.
\newblock Princeton University Press, 1967.

\bibitem{leibniz1703}
Gottfried Wilhelm Leibniz.
\newblock Explication de l'arithm\'etique binaire.
\newblock {\em M\'emoires de l'Acad\'emie Royale des Sciences}, 1703.

\end{thebibliography}

\appendix

\section{Full King Wen Binary Table}

\begin{center}
\small
\begin{tabular}{cc|cc|cc|cc}
\toprule
KW & Bin & KW & Bin & KW & Bin & KW & Bin \\
\midrule
1 & 63 & 17 & 25 & 33 & 60 & 49 & 29 \\
2 & 0 & 18 & 38 & 34 & 15 & 50 & 46 \\
3 & 17 & 19 & 3 & 35 & 40 & 51 & 9 \\
4 & 34 & 20 & 48 & 36 & 5 & 52 & 36 \\
5 & 23 & 21 & 41 & 37 & 53 & 53 & 52 \\
6 & 58 & 22 & 37 & 38 & 43 & 54 & 11 \\
7 & 2 & 23 & 32 & 39 & 20 & 55 & 13 \\
8 & 16 & 24 & 1 & 40 & 10 & 56 & 44 \\
9 & 55 & 25 & 57 & 41 & 35 & 57 & 54 \\
10 & 59 & 26 & 39 & 42 & 49 & 58 & 27 \\
11 & 7 & 27 & 33 & 43 & 31 & 59 & 50 \\
12 & 56 & 28 & 30 & 44 & 62 & 60 & 19 \\
13 & 61 & 29 & 18 & 45 & 24 & 61 & 51 \\
14 & 47 & 30 & 45 & 46 & 6 & 62 & 12 \\
15 & 4 & 31 & 28 & 47 & 26 & 63 & 21 \\
16 & 8 & 32 & 14 & 48 & 22 & 64 & 42 \\
\bottomrule
\end{tabular}
\end{center}

Binary values represent the hexagram as a 6-bit integer, with bit 0 as the bottom line.

\end{document}
